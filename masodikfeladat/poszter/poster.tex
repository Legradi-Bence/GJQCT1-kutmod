\documentclass[final]{beamer}

% A0 poster with beamer
\usepackage[size=a0,orientation=portrait,scale=1.2]{beamerposter}

\usetheme{default}
\usecolortheme{default}
\setbeamertemplate{navigation symbols}{} % no nav buttons

\usepackage[utf8]{inputenc}
\usepackage[T1]{fontenc}
\usepackage[english]{babel}
\usepackage{graphicx}
\usepackage{booktabs}

\title{Predicting Winning Teams in League of Legends \\ Using a Random Forest Classifier}
\author{Légrádi Bence}
\institute{December 8, 2025}

\begin{document}
\begin{frame}[t]

\begin{columns}[t,totalwidth=\textwidth]

% ========== LEFT COLUMN ==========
\begin{column}{0.32\textwidth}

\begin{block}{Abstract}
This project aims to predict the winning team in League of Legends (LoL)
matches using a Random Forest classifier. By analyzing various in-game
statistics, we identify key features that influence match outcomes and
evaluate the model's performance. The classifier achieves high accuracy and
provides insight into which aspects of the game are most strongly associated
with victory.
\end{block}

\begin{block}{Introduction}
League of Legends is a popular multiplayer online battle arena (MOBA) game
where two teams of five players compete to destroy the enemy Nexus.
Throughout a match, many statistics are recorded, such as kills, objectives,
towers, and other actions. These statistics reflect the state of the game
and can be used to infer which team is likely to win.

The goal of this work is to investigate how well we can predict the final
winner of a match from in-game features, using a Random Forest classifier
trained on historical ranked game data.
\end{block}

\begin{block}{Dataset}
\begin{itemize}
    \item Source: Kaggle League of Legends ranked games dataset.
    \item Contains a wide range of in-game statistics: kills, characters,
          objectives taken, towers, inhibitors, and more.
    \item Target variable: winning team, encoded as 1 for Team 1
          and 2 for Team 2.
    \item Identifier columns were removed, as they do not provide useful
          information for the model.
\end{itemize}

The distribution of winning teams is relatively balanced:
Team 1 wins 51.2\% of matches, Team 2 wins 48.8\%. This balance is
important for training a fair and unbiased model.

\begin{center}
\IfFileExists{diagrams/gyozelmek_eloszlasa.png}{%
  \includegraphics[width=0.9\linewidth]{diagrams/gyozelmek_eloszlasa.png}%
}{%
  \fbox{Winning Team Distribution Figure}%
}
\end{center}
\end{block}

\end{column}

% ========== MIDDLE COLUMN ==========
\begin{column}{0.36\textwidth}

\begin{block}{Game Length Analysis}
We also examined the length of the matches. The distribution of game length
shows that most games last between 30 and 40 minutes. This suggests that
the majority of ranked games are decided in a typical mid- to late-game
time window.

\begin{center}
\IfFileExists{diagrams/meccs_hossza_eloszlas.png}{%
  \includegraphics[width=0.9\linewidth]{diagrams/meccs_hossza_eloszlas.png}%
}{%
  \fbox{Game Length Distribution Figure}%
}
\end{center}
\end{block}

\begin{block}{Model Training}
\begin{itemize}
    \item The dataset was split into a training set (80\%) and a test set (20\%).
    \item A Random Forest classifier was trained on the training set.
    \item The model uses a collection of decision trees whose predictions are
          combined to obtain robust and accurate results.
\end{itemize}

Random Forests are well-suited for this task because they can handle many
features, capture non-linear relationships, and provide feature importance
scores that help interpret which variables matter most for the predictions.
\end{block}

\begin{block}{Results}
The performance of the model was evaluated on the test set using accuracy
as the primary metric.

\begin{itemize}
    \item Test accuracy: \textbf{97\%}.
    \item Out of 2000 test matches, the model correctly predicted
          1944 outcomes.
\end{itemize}

The confusion matrix illustrates the number of correctly and incorrectly
classified matches.

\begin{center}
\IfFileExists{diagrams/confusion_matrix.png}{%
  \includegraphics[width=0.9\linewidth]{diagrams/confusion_matrix.png}%
}{%
  \fbox{Confusion Matrix Figure}%
}
\end{center}

To further evaluate the classifier, the ROC curve and the Area Under the
Curve (AUC) were computed. The model achieves an AUC of 0.99, indicating
excellent capability to distinguish between the two classes.

\begin{center}
\IfFileExists{diagrams/roc_gorbe.png}{%
  \includegraphics[width=0.9\linewidth]{diagrams/roc_gorbe.png}%
}{%
  \fbox{ROC Curve Figure}%
}
\end{center}
\end{block}

\end{column}

% ========== RIGHT COLUMN ==========
\begin{column}{0.32\textwidth}

\begin{block}{Feature Importance}
To understand which features contributed most to the predictions, we
analyzed the feature importance scores provided by the Random Forest
classifier.

The top features include:
\begin{itemize}
    \item Total number of tower kills.
    \item Total number of inhibitor kills.
    \item Other objective-related statistics (such as dragons and barons taken).
\end{itemize}

\begin{center}
\IfFileExists{diagrams/fontos_featureok.png}{%
  \includegraphics[width=0.9\linewidth]{diagrams/fontos_featureok.png}%
}{%
  \fbox{Top Feature Importances Figure}%
}
\end{center}

These results are intuitive: destroying towers and inhibitors is crucial for
winning, because towers provide defense and inhibitors, when destroyed,
allow a team to spawn stronger minions and exert more pressure on lanes.
A team that destroys more towers and inhibitors gains a significant strategic
advantage that often leads to victory.

Interestingly, champion picks do not appear among the most important
features. This suggests that the outcome of a game is driven more by
in-game actions and strategies than by the specific champions chosen,
indicating that champions are relatively well-balanced with respect to
match outcome.
\end{block}

\begin{block}{Conclusion}
The Random Forest classifier effectively predicts the winning team in
League of Legends ranked matches based on in-game statistics. Key features
such as tower and inhibitor kills play a major role in determining match
outcomes. The model's high accuracy and strong evaluation metrics show its
potential for practical applications in game analysis and strategic decision
support.

Future work could explore additional features, alternative modeling
techniques, or time-dependent representations of the match (for example,
predicting win probability at different points in time), in order to further
improve prediction performance and provide more detailed insights into game
dynamics.
\end{block}

\end{column}

\end{columns}

\end{frame}
\end{document}
